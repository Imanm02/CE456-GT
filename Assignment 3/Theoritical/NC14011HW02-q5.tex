برای پیدا کردن یک استراتژی متقارن تعادل نش در یک بازی دو نفره با ماتریس بازدهی R ، می‌دانیم که اگر 
$x^{*}$
استراتژی متقارن تعادل نش باشد، آنگاه باید داشته باشیم:

\[ x^{*T}Rx* \geq x^{T}Rx* \]

برای هر بردار x مثبت. اگر فرض کنیم که همه درایه‌های
$x^{*}$
مثبت هستند، آنگاه این برابری تنها وقتی برقرار است که
$x^{*}$
یک بردار ویژه ماتریس R باشد که به آن مقدار ویژه بزرگ‌ترین ممکن
$(\lambda_{max})$
تعلق دارد.

برای یافتن $x^{*}$، می‌توانیم از تعریف بردارهای ویژه استفاده کنیم:

\[ Rx* = \lambda_{max} x* \]

و از آنجا که می‌دانیم که
$x^{*}$
یک توزیع احتمال است (یعنی مجموع درایه‌های آن برابر با یک است)، می‌توانیم آن را به صورت زیر محاسبه کنیم:

\[ x* = \lambda_{max}^{-1}Rx* \]

با توجه به مقدار ویژه بیشینه
$\lambda_{max}$
، می‌توانیم این رابطه را برای
$x^{*}$
استفاده کنیم. از طرفی، اگر چنین
$x^{*}$
وجود داشته باشد که همه درایه‌های آن مثبت باشند، آنگاه آن یکتا خواهد بود، زیرا مقدار ویژه بیشینه یکتا است.

چنین تعادلی وجود خواهد داشت اگر و تنها اگر مقدار ویژه بیشینه ماتریس R مثبت باشد و بردار ویژه متناظر با آن تمام درایه‌های مثبت داشته باشد. 

اگر یک استراتژی متقارن تعادل نش
$x^{*}$
با تمام درایه‌های مثبت وجود داشته باشد، آنگاه آن یک استراتژی پایدار تکاملی خواهد بود اگر و تنها اگر برای هر استراتژی y مختلط مثبت که
$y \neq x^{*}$
داریم:

\[ y^{T}Ry < x^{*T}Rx* \]

یا به عبارت دیگر، اگر و تنها اگر برای هر استراتژی y مختلط مثبت دیگر، بازیکن با استفاده از استراتژی
$x^{*}$
بیشتر از بازیکن با استفاده از استراتژی y برنده می‌شود.
