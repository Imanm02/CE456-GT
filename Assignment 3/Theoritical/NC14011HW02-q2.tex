اثبات می‌کنیم در هر بازی متقارن، اگر
$
a_{i, i} > a_{i, j}
$
به ازای تمامی
$
i \neq j
$
برقرار باشد، آنگاه استراتژی خالص i یک استراتژی پایدار تکاملی است.

در حالت کلی، برای اثبات اینکه یک استراتژی خالص در یک بازی یک استراتژی پایدار تکاملی

\lr{(Evolutionary Stable Strategy - ESS)}
است، باید دو شرط را بررسی کنیم:


۱. استراتژی خالص
\lr{i}
باید برنده استراتژی خالص
\lr{j}
را وقتی بقیه بازیکنان استراتژی
\lr{i}
را انتخاب کرده باشند، برابر یا بیشتر از استراتژی
\lr{j}
کند.

۲. اگر استراتژی
\lr{i}
و
\lr{j}
هر دو به یک اندازه برنده باشند، آنگاه استراتژی
\lr{i}
باید برنده استراتژی
\lr{j}
را وقتی بقیه بازیکنان استراتژی
\lr{j}
را انتخاب کرده باشند، بیشتر از استراتژی
\lr{j}
کند.

بنابراین، برای اثبات این موضوع برای یک بازی متقارن، کافیست دو شرط فوق را برای ماتریس بازی بررسی کنیم. اگر ماتریس بازی به گونه ای باشد که 
$a_{i, i} > a_{i, j}$
برای همه 
$i ≠ j$
، بنابراین این دو شرط برقرار هستند و استراتژی
\lr{i}
یک استراتژی پایدار تکاملی
\lr{(ESS)}
است.

همچنین این موضوع به طور کلی نشان می دهد که در یک بازی متقارن، استراتژی 
\lr{i}
که بیشترین پی‌آف
\lr{(payoff)}
را به بازیکن می‌دهد (یعنی 
$a_{i, i}$
بیشترین مقدار را دارد) یک استراتژی پایدار تکاملی است.


برای مثال در این بازی:
\begin{center}	
	\begin{array}{c|cc}
		& 1 & 2 \\ \hline
		1 & a_{1, 1} & a_{1, 2} \\
		2 & a_{2, 1} & a_{2, 2} \\
	\end{array}
\end{center}	

این ماتریس بازی است که دو استراتژی خالص دارد: استراتژی 1 و استراتژی 2 . 
در یک بازی متقارن،
$a_{i, j} = a_{j, i}$
برقرار است. اگر شرایط مساله برقرار باشد که 
$a_{1, 1} > a_{1, 2}$ و $a_{2, 2} > a_{2, 1}$
، بنابراین هر دو استراتژی پایدار تکاملی
\lr{(ESS)}
هستند.

برای مثال، فرض کنید
$a_{1, 1} = 3$، $a_{1, 2} = 2$، $a_{2, 1} = 2$ و $a_{2, 2} = 3$
. 
در این حالت، هر دو استراتژی
\lr{(ESS)}
هستند. چرا که اگر همه بازیکنان استراتژی 1 را انتخاب کنند، هر بازیکنی که به استراتژی 2 تغییر کند، پاداش کمتری دریافت می‌کند و برعکس.

