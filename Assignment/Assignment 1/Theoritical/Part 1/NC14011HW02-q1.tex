اثبات تئوری نش


در این‌جا با فرض این‌که basics گیم‌تئوری، از جمله تعریف
$$pure-strategy$$
و
$$mixed-strategy
\hspace*{0.1cm}
Nash
\hspace*{0.1cm}
equilibrium$$
را می‌دانیم، به اثبات تئوری نش می‌پردازیم و از مثال زیر نیز بهره می‌بریم.

تعریف قضیه‌ی نش: مجموعه
$(S, F)$
را به عنوان بازی با n بازیکن در نظر بگیرید، که در آن
$S_i$
مجموعه راهبردها برای خرید بازیکن i است،
$S = S_1 * S_2 * ... S_n$
مجموعه‌ای از فضای راهبرد آن است و
$F(x)$
تابع بهره‌مندی آن است،
$X_-i$
را به عنوان فضای راهبرد همه‌ی بازیکنان به جز بازیکن i در نظر بگیرید. هر بازیکن به ازای هر i عضو مجموعه اعداد صحیح، راهبرد
$X_i$
را انتخاب کند، پروفایل راهبرد آن به صورت
$x = x_1, ..., x_n$
و تابع بهره‌مندی آن به صورت
$F(x_i)$
نتیجه داده می‌شود. قابل ذکر است که تابع بهره‌مندی به نمای راهبرد انتخابی وابسته است. به عنوان مثال، در راهبرد انتخاب شده توسط بازیکن i و همچنین راهبردهای انتخاب شده توسط تمام بازیکنان دیگر. نمای راهبرد
$x^* \in S$
یک تعادل نش
$(NE)$
است اگر هیچ انحراف یک‌سویی در راهبرد توسط هر بازیکن واحد با یکی دیگر از بازیکنان سودآور نمی‌باشد.
یعنی:
$$\forall i, x_i  \in S_i , x_i \neq x_i^* : f_i(x_i^* , x_{-i}^*) \geq f_i(x_i , x_{-i}^*)$$

یک بازی می‌تواند یا راهبرد محض یا تعادل نش ترکیبی باشد،(در تعریف اخیر راهبردی محض آن است که به صورت تصادفی با فراوانی ثابت انتخاب شده‌است). نش نشان داد که در صورتی به ما اجازه راهبرد ترکیب شده بدهند، سپس هر بازی با تعداد محدودی از بازیکنان که در آن هر بازیکن می‌تواند به صورت غیر محدود از میان بسیاری از راهبرد‌های کامل که حداقل یک تعادل نش می‌باشد انتخاب کند. وقتی نابرابری اکید در بالا نگه می‌دارد برای تمام بازیکنان و تمام راهبرد‌های جایگزین امکان‌پذیر است، سپس تعادل طبقه‌بندی شده به عنوان یک تعادل دقیق نش می‌باشد. اگر در عوض، برای برخی از بازیکنان، برابری دقیقی بین x و بعضی از راهبرد‌های مجموعه S وجود دارد. سپس تعادل به عنوان یک تعادل طبقه‌بندی شده ضعیفی از نش می‌باشد.


اثبات قضیه‌ی نش:

استفاده از قضیهٔ نقطهٔ ثابت کاکوتانی 
$(Kakutani's fixed-point theorem)$ 
می‌تواند به عنوان یکی از روش‌هایی که برای اثبات وجود تعادل نش در بازی‌های غیرخطی استفاده می‌شود، به کار گرفته شود. در اینجا، ابتدا به تعریف قضیهٔ نقطهٔ ثابت کاکوتانی پرداخته و سپس نحوهٔ استفاده از آن برای اثبات وجود تعادل نش را شرح می‌دهیم.

قضیهٔ نقطهٔ ثابت کاکوتانی:
فرض کنید $S$ یک مجموعهٔ خاموش و منحصر به فرد باشد و $F : S \rightarrow \Delta(S)$ یک نگاشت چگال باشد. در این صورت، حداقل یک نقطهٔ ثابت برای $F$ وجود دارد.

برای استفاده از قضیهٔ نقطهٔ ثابت کاکوتانی در اثبات تعادل نش، ابتدا باید بازی و استراتژی‌های بازیکنان را تعریف کنیم. فرض کنید $N$ بازیکن داریم و هر بازیکن $i$ یک مجموعهٔ استراتژی $S_i$ دارد. همچنین، فرض کنید
$$u_i : S_1 \times \dots \times S_N \rightarrow \mathbb{R}$$ 
تابع منفعت بازیکن $i$ باشد. با استفاده از تعریف‌های بالا، بازی معین است.

حال، فرض کنید $F_i : S_1 \times \dots \times S_N \rightarrow \Delta(S_i)$ نگاشت چگال است که بازیکن $i$ را به استراتژی‌ای از خود هدایت می‌کند. به عبارت دیگر، بازیکن $i$ با انتخاب استراتژی $s_i \in S_i$، با احتمال $F_i(s_1, \dots, s_N)(s_i)$ به استراتژی $s_i$ تغییر می‌دهد. سپس، فرض کنید $u_i(s_1, \dots, s_N)$ منفعت بازیکن $i$ باشد.

حال با توجه به تعریف‌های بالا، به دنبال تعادل نش در بازی هستیم. یعنی برای هر بازیکن $i$ و هر استراتژی $s_i \in S_i$، با احتمال $p_i(s_1, \dots, s_N) \geq 0$ باید داشته باشیم:
$$u_i(s_1, \dots, s_N) \geq u_i(s_1, \dots, s_{i-1}, t_i, s_{i+1}, \dots, s_N)$$

برای هر $t_i \in S_i$. به عبارت دیگر، هیچ بازیکنی نمی‌تواند با تغییر استراتژی خود، منفعت خود را افزایش دهد. به این تعریف، تعریف تعادل نش گفته می‌شود.

برای اثبات وجود تعادل نش، ابتدا فرض کنید $F_i$ ها چنین نگاشت‌هایی نباشند که مستقیما به تعادل نش منجر شوند. به عبارت دیگر، فرض کنید هیچ $s_1, \dots, s_N$ ای وجود ندارد که برای هر بازیکن $i$ و هر $t_i \in S_i$ داشته باشیم:
$$u_i(s_1, \dots, s_N) < u_i(s_1, \dots, s_{i-1}, t_i, s_{i+1}, \dots, s_N)$$
حال، می‌توانیم قضیهٔ نقطهٔ ثابت کاکوتانی را در اینجا به کار بگیریم. برای هر بازیکن $i$، نگاشت $F_i : S_1 \times \dots \times S_N \rightarrow \Delta(S_i)$ را در نظر می‌گیریم. با توجه به تعریف قضیهٔ نقطهٔ ثابت کاکوتانی، حداقل یک نقطهٔ ثابت برای $F_i$ وجود دارد. به عبارت دیگر، یک بردار احتمال $p_i \in \Delta(S_i)$ وجود دارد به طوری که $p_i = F_i(p_1, \dots, p_N)$.

سپس، فرض کنید $u_i(p_1, \dots, p_N)$ حداکثر منفعت بازیکن $i$ باشد. با توجه به فرض قبلی، می‌توان نشان داد که برای هر $t_i \in S_i$، داریم:
$$u_i(p_1, \dots, p_N) \geq u_i(p_1, \dots, p_{i-1}, t_i, p_{i+1}, \dots, p_N)$$
حال فرض کنید $p_i$ را با $t_i$ جایگزین کنیم و $p_i' = F_i(p_1, \dots, p_{i-1}, t_i, p_{i+1}, \dots, p_N)$ را برای $p_i$ قرار دهیم. با توجه به اینکه $p_i = F_i(p_1, \dots, p_N)$، داریم:
$$u_i(p_1, \dots, p_N) \geq u_i(p_1, \dots, p_{i-1}, p_i', p_{i+1}, \dots, p_N)$$
حال فرض کنید $p_i'$ را با $t_i$ جایگزین کنیم و $p_i'' = F_i(p_1, \dots, p_{i-1}, t_i, p_{i+1}, \dots, p_N)$ را برای $p_i'$ قرار دهیم. با توجه به فرض قبلی، داریم:
$$u_i(p_1, \dots, p_N) \geq u_i(p_1, \dots, p_{i-1}, p_i', p_{i+1}, \dots, p_N) \geq u_i(p_1, \dots, p_{i-1}, p_i'', p_{i+1}, \dots, p_N)$$
با این روش ادامه دهیم و $p_i''' = F_i(p_1, \dots, p_{i-1}, t_i, p_{i+1}, \dots, p_N)$ را برای $p_i''$ قرار دهیم، می‌توانیم نشان دهیم که داریم:
$$u_i(p_1, \dots, p_N) \geq u_i(p_1, \dots, p_{i-1}, p_i', p_{i+1}, \dots, p_N) \geq \dots \geq u_i(p_1, \dots, p_{i-1}, p_i^{(k)}, p_{i+1}, \dots, p_N)$$
در نتیجه، مجموعه $p_i, p_i', p_i'', \dots, p_i^{(k)}$ یک زنجیرهٔ کوچکتر مساوی با $k+1$ اعضا در $\Delta(S_i)$ است. با توجه به اینکه $S_i$ محدود و مجموعه $\Delta(S_i)$ فضای مرتب است، طبق اصل زنجیرهٔ کوچکترین مجموعه، مجموعهٔ $p_i, p_i', p_i'', \dots, p_i^{(k)}$ یک نقطهٔ ثابت در $F_i$ است، به عبارت دیگر، داریم:
$$p_i^{(k+1)} = F_i(p_1, \dots, p_{i-1}, t_i, p_{i+1}, \dots, p_N) = p_i^{(k)}$$
حال به این نتیجه می‌رسیم که نقطهٔ ثابت $p^* = (p_1^, \dots, p_N)$ وجود دارد. برای نشان دادن این موضوع، مجموعهٔ $\Delta(S_1) \times \dots \times \Delta(S_N)$ را در نظر بگیرید. این مجموعه یک فضای مرتب کامل است. برای هر $i \in {1, \dots, N}$، تابع $F_i$ یک تابع متناوب بر روی مجموعهٔ $\Delta(S_1) \times \dots \times \Delta(S_N)$ است. به عبارت دیگر، برای هر 
$$x \in \Delta(S_1) \times \dots \times \Delta(S_N)$$
داریم:
$$F_i(F_j(x)) = F_j(F_i(x))$$
حال با توجه به قضیهٔ Kakutani مجموعهٔ نقطهٔ ثابت تابع متناوب معین بر روی یک فضای مرتب کامل، غیرتهی است. بنابراین، یک نقطهٔ ثابت $p^* = (p_1^, \dots, p_N)$ وجود دارد که برای آن داریم:
$$p_i^* = F_i(p_1^, \dots, p_N)$$
در نتیجه، نقطهٔ $p^* = (p_1^, \dots, p_N)$ تعادل نش بازی است. زیرا هیچ بازیکنی نمی‌تواند با تغییر استراتژی خود، امتیاز خود را به طور غیرقابل تغییری بهبود ببخشد. به عبارت دیگر، هیچ بازیکنی نمی‌تواند با انتخاب استراتژی متفاوت، امتیاز خود را از $u_i(p_1^, \dots, p_N)$ به $u_i(p_1', \dots, p_N')$ بهبود ببخشد.

حال باید نشان داد که نقطهٔ ثابت $p^$ تکرارناپذیر است. فرض کنید که دو نقطهٔ ثابت $p^$ و $q^*$ وجود دارد، به عبارت دیگر:
$$p_i^* = F_i(p_1^, \dots, p_N) \quad \text{و} \quad q_i^* = F_i(q_1^, \dots, q_N)$$
حال برای هر $i \in {1, \dots, N}$، داریم:
$$p_i^* = F_i(p_1^, \dots, p_{i-1}^, q_i^, p_{i+1}^, \dots, p_N*)$$
بنابراین:
$$u_i(p_1^, \dots, p_{i-1}^, q_i^, p_{i+1}^, \dots, p_N) \leq u_i(p_1^, \dots, p_{i-1}^, p_i^, p_{i+1}^, \dots, p_N)$$
اما به دلیل تعادل نش، داریم:
$$u_i(p_1^, \dots, p_{i-1}^, p_i^, p_{i+1}^, \dots, p_N) \leq u_i(q_1^, \dots, q_{i-1}^, q_i^, q_{i+1}^, \dots, q_N)$$
بنابراین:
$$u_i(p_1^, \dots, p_{i-1}^, q_i^, p_{i+1}^, \dots, p_N) \leq u_i(q_1^, \dots, q_{i-1}^, q_i^, q_{i+1}^, \dots, q_N)$$
به طور مشابه، برای هر $i \in {1, \dots, N}$، داریم:
$$u_i(q_1^, \dots, q_{i-1}^, q_i^, q_{i+1}^, \dots, q_N) \leq u_i(p_1^, \dots, p_{i-1}^, q_i^, p_{i+1}^, \dots, p_N)$$
بنابراین:
$$u_i(p_1^, \dots, p_{i-1}^, q_i^, p_{i+1}^, \dots, p_N) = u_i(q_1^, \dots, q_{i-1}^, q_i^, q_{i+1}^, \dots, q_N)$$
از طرفی، برای هر $i \in {1, \dots, N}$، داریم:
$$p_i^* = F_i(p_1^, \dots, p_{i-1}^, q_i^, p_{i+1}^, \dots, p_N) = q_i^$$

بنابراین، هر دو نقطهٔ ثابت $p^$ و $q^$ برابر هستند و در نتیجه، نقطهٔ ثابت یکتایی وجود دارد. این نقطهٔ ثابت، به عنوان یک تعادل نش در نظریهٔ بازی، شناخته می‌شود.

با استفاده از الگوریتم Kakutani's fixed-point می‌توان تعادل نش را برای هر بازی با مجموعه‌ای از نمایندگان و فضای اقداماتشان به دست آورد. برای این کار، ابتدا باید تابعی به نام تابع پیشنهادات را به عنوان تابعی که ورودی‌هایی مانند پیشنهادات بازیکنان را دریافت می‌کند و خروجی‌ای مانند پیشنهادات جدید بازیکنان را تولید می‌کند، تعریف کرد. سپس با استفاده از این تابع، یک سیستم معادلات نامعادل را حل می‌کنیم تا به نقطهٔ ثابت یا تعادل نش برسیم.

در نهایت، باید توجه داشت که این روش برای بعضی از بازی‌ها ممکن است به نقطهٔ ثابت نرسد و در این صورت، باید از روش‌های دیگری مانند روش اعمال تصادفی یا روش تکرار باشگاهی استفاده کرد.

با فرض وجود $p^$ به عنوان یک نقطه ثابت، حالتی که $q$ به صورت 
$$q=(1-\epsilon)p + \epsilon\hat{q}$$
 با $\epsilon>0$ و $\hat{q}\in\Delta_{n-1}$ تعریف شده است را در نظر می‌گیریم. با جایگذاری $q$ در نامعادلهٔ $(1)$، داریم:
$$p(u_i(p^,q))\geq p(u_i(p,(1-\epsilon)p+\epsilon\hat{q})).$$
در اینجا، $p(u_i(p,q))$ بیانگر بهرهٔ بازیکن $i$ در حالت تعادل نش است. با توجه به تعریف $p^$ به عنوان نقطهٔ ثابت، داریم:
$$p(u_i(p,q))\geq p(u_i(p,(1-\epsilon)p+\epsilon\hat{q}))= \sum_{j=1}^{n-1}p_j^u_i(p_i,\hat{q})+\epsilon(u_i(p_i,\hat{q})-u_i(p,\hat{q})).$$
حال، می‌توانیم به سادگی برای حالتی که $\epsilon$ بسیار کوچک است، بالا و پایین نامعادلهٔ فوق را در نظر بگیریم و آن‌ها را به ازای 
$\epsilon\rightarrow 0$
با یکدیگر برابر بگیریم:
$$p(u_i(p,q))\geq \sum_{j=1}^{n-1}p_j^u_i(p_i,\hat{q})+(u_i(p_i,\hat{q})-u_i(p,\hat{q})).$$
از طرفی، می‌توانیم $q$ را به صورت $q=(1-\epsilon)\hat{p}+\epsilon p^$ با 
$\hat{p}^\in\Delta_{n-1}$
تعریف کنیم. با جایگذاری $q$ در نامعادلهٔ $(2)$، داریم:
$$\hat{p}^(u_i(\hat{p},q))\leq \hat{p}^(u_i(\hat{p},(1-\epsilon)\hat{p}+\epsilon p)).$$
با توجه به تعریف $q$، داریم:

حال، با ترکیب کردن نامعادلهٔ بالا با نامعادلهٔ $(3)$ و جمع کردن طرفین، داریم:

$$&\sum_{i=1}^n p_i^*(u_i(p^*,q)) + \sum_{i=1}^n \hat{p}_i^*(u_i(\hat{p}^*,q)) \\
&\qquad \geq $$
$$\sum_{i=1}^{n-1}\left(p_i^*(u_i(p_i^*,\hat{q}))
+\hat{p}_i^*(u_i(\hat{p}_i^*,p^*))\right)$$
$$&\qquad\qquad\qquad+\epsilon\left(\sum_{i=1}^{n-1}(u_i(p_i^*,\hat{q}))-u_i(p^*,\hat{q})+\sum_{i=1}^{n-1}(u_i(\hat{p}_i^*,p^*))-u_i(\hat{p}^*,p^*)\right).
\end{aligned}$$
حال، با توجه به این که $\hat{q},p^*\in \Delta_{n-1}$، می‌توان نشان داد که دو جملهٔ آخر کوچکتر مساوی صفر هستند:
$$\begin{aligned}
\sum_{i=1}^{n-1}(u_i(p_i^*,\hat{q}))-u_i(p^*,\hat{q}) &\leq \max_{i\in[n-1]}\left(u_i(p_i^*,\hat{q}))-u_i(p^*,\hat{q})\right)\\
&\leq \max_{i,j\in[n-1]}\left|u_i(p_j^*,\hat{q})-u_i(p^*,\hat{q})\right|\\
&\leq \max_{i,j\in[n-1]}\left|u_i(p_j^*,q)-u_i(p^*,q)\right|+\max_{i\in[n-1]}\left|u_i(p_i^*,\hat{q})-u_i(p^*,\hat{q})\right|\\
&\leq 2\delta.
\end{aligned}$$
مشابهاً، داریم:
$$\sum_{i=1}^{n-1}(u_i(\hat{p}_i^*,p^*))-u_i(\hat{p}^*,p^*)\leq 2\delta.$$
با توجه به این نتایج، داریم:
$$\sum_{i=1}^n p_i^*(u_i(p^*,q)) + \sum_{i=1}^n \hat{p}_i^*(u_i(\hat{p}^*,q)) \geq \sum_{i=1}^{n-1}\left(p_i^*(u_i(p_i^*,\hat{q}))+\hat{p}_i^*(u_i(\hat{p}_i^*,p^*))\right)-4\delta.$$
در نتیجه، با توجه به نامعادلهٔ $(4)$ می‌رسیم:
$$\sum_{i=1}^n p_i^(u_i(p,q)) + \sum_{i=1}^n \hat{p}i^(u_i(\hat{p},q)) \geq \sum{i=1}^{n-1}\left(p_i^(u_i(p_i^,\hat{q}))+\hat{p}i^(u_i(\hat{p}_i^,p^))\right)-4\delta.$$
حال، با توجه به این که $\hat{p}^$ و $p^*$ دو نقطهٔ دلخواه در $\Delta{n-1}$ هستند، می‌توانیم نتیجه بگیریم که تابع $F(p,q)=\sum_{i=1}^n p_i(u_i(p,q))$ یک تابع مستقل از متغیر $q$ است. به عبارت دیگر، برای هر $q\in\Delta_{n-1}$، تعادل نش ما $p^(q)$ باید برابر با یک $p^$ ثابت باشد که بیانگر تعادل نش کلی است. در نتیجه، می‌توانیم برای هر $q\in\Delta_{n-1}$، تابع $F(p^(q),q)$ را برابر با $F(p^,q)$ در نظر بگیریم. با توجه به نامعادلهٔ $(4)$، داریم:
$$F(p^(q),q) + F(\hat{p}^(q),q) \geq 2\sum_{i=1}^{n-1}\left(p_i^(q)u_i(p_i^(q),\hat{q}))+\hat{p}i^(q)u_i(\hat{p}_i^(q),p)\right)-4\delta.$$
با استفاده از نامعادلهٔ $(2)$، داریم:
$$F(p^(q),q) + F(\hat{p}^(q),q) \geq 2F(p^(q),\hat{q}) - 4\delta.$$
از طرفی، با توجه به نامعادلهٔ $(1)$، برای هر $q\in\Delta{n-1}$، داریم:
$$F(p^(q),q)\geq F(p^,q)-\delta.$$
بنابراین، داریم:
$$F(p^*,q) + F(\hat{p}^*(q),q) &\geq 2F(p^*(q),q) - 4\delta \\
&\geq 2(F(p^*,q)-\delta)-4\delta \\
&= 2F(p^*,q) - 6\delta.
\end{aligned}$$

از طرفی، با توجه به نامعادلهٔ $(3)$، داریم:
$$F(p,q)+F(q,p)\geq 2V-\epsilon.$$
از این دو نامعادله، داریم:
$$2V-\epsilon &\leq F(p^*,q)+F(q,p^*) \\
&\leq 2F(p^*,q)+2F(\hat{p}^*(q),q)-2\delta \\
&\leq 2F(p^*,q)+2F(p^*,\hat{q})-12\delta \\
&\leq 2V - 14\delta - \epsilon.
\end{aligned}$$

بنابراین، با جمع کردن نامعادلهٔ بالا، داریم:
$$\delta \leq \frac{\epsilon}{14}.$$
این نشان می‌دهد که در هر تعادل نش، ما می‌توانیم تضمین کنیم که فاصله بین پاسخ بهینه و پاسخ تعادلی حداکثر $\frac{\epsilon}{14}$ است. با توجه به این که $\epsilon$ مقدار دلخواه کوچک است، می‌توانیم نتیجه بگیریم که فاصلهٔ بین پاسخ بهینه و پاسخ تعادلی در حداکثر $\frac{1}{n}$ است. به عبارت دیگر، در هر تعادل نش، پاسخ تعادلی بهینهٔ نزدیکی به پاسخ بهینه دارد.

حال به اثبات نشان می‌دهیم که این تعادل نش در بازی‌های ماتریسی به دست می‌آید.

فرض کنید $G$ یک بازی ماتریسی با $n$ بازیکن و $m_1,\ldots,m_n$ شمارهٔ اکشن‌های بازیکنان باشد. ماتریس پرداخت‌ها را با $A=[a_{ij}]$ نشان می‌دهیم. برای هر بازیکن $i$، مجموعهٔ اکشن‌های ممکن او را با $S_i$ نشان می‌دهیم.

تابع منفعت بازیکن $i$ را با $u_i:S_1\times\cdots\times S_n \to \mathbb{R}$ نشان می‌دهیم. تعریف می‌کنیم:
$$V_i(p)=\max_{q\in S_{-i}}\sum_{s\in S}p(s)u_i(s,q),$$
که در آن $S_{-i}=S_1\times\cdots\times S_{i-1}\times S_{i+1}\times\cdots\times S_n$ و $p(s)$ احتمال طبیعی برای اکشن $s\in S$ است.

برای هر بازیکن $i$ و هر $q\in S_{-i}$، ماتریس پرداخت مستقیم $P_i(q)$ به شکل زیر تعریف می‌شود:
$$(P_i(q))_{s,t}=\begin{cases}1, &\text{اگر }u_i(s,q)\geq u_i(t,q),\ 0, &\text{در غیر این صورت.}\end{cases}$$

در واقع، $P_i(q)$ ماتریسی است که در آن می‌خواهیم اکشن‌هایی را که برای بازیکن $i$ باعث به دست آمدن پاداش بیشتر می‌شوند را پیدا کنیم.

برای هر بازیکن $i$، ماتریس پرداخت موزون $M_i(q)$ به شکل زیر تعریف می‌شود:
$$M_i(q)=P_i(q)A.$$
به عبارت دیگر، ما ابتدا برای بازیکن $i$، ماتریس پرداخت مستقیم را به دست می‌آوریم و سپس آن را در ماتریس پرداخت کل ضرب می‌کنیم. این ماتریس، ماتریس پر 

در مرحلهٔ آخر اثبات، می‌توانیم نشان دهیم که برد بازی روی مجموعه‌ی تعادل نش واقع می‌شود. اگر یکی از بازیکنان در تعادل نش بازی کرده باشد، آنگاه با توجه به تعریف تعادل نش، هیچ یک از بازیکنان نمی‌توانند با انتخاب استراتژی دیگری پاداش خود را افزایش دهند، به عبارت دیگر، هیچ یک از بازیکنان نمی‌توانند با تغییر استراتژی‌هایشان بیشتر از پاداشی که در تعادل نش کسب می‌کنند، به دست آورند.

از آنجا که هر استراتژی در مجموعه‌ی تعادل نش با احتمالی غیر صفر انتخاب می‌شود، پس پاداش حاصل از بازی روی مجموعه‌ی تعادل نش برابر با پاداش حاصل از بازی روی استراتژی‌های تعادل نش است. به عبارت دیگر، برد بازی برابر با پاداشی است که بازیکنان در تعادل نش کسب می‌کنند. این نتیجه نشان می‌دهد که مجموعه‌ی تعادل نش نه تنها وجود دارد، بلکه برای هر بازی به شکلی که شرایط بالا برقرار باشد، حداقل یک تعادل نش وجود دارد.

بنابراین، اثبات کردیم که با استفاده از تعریف تعادل نش و به کمک مبدأ Kakutani's fixed-point در هر بازی با مجموعه‌ی استراتژی متناهی و مستقل از زمان و با پاداش‌های متناهی، حداقل یک تعادل نش وجود دارد.

لازم به ذکر است که وجود تعادل نش به دلیل تعداد بی‌نهایت از ترکیب‌های مختلف استراتژی‌ها و پاداش‌ها در بازی‌های پیچیده، ممکن است در برخی موارد بسیار دشوار باشد. همچنین، در برخی بازی‌ها، ممکن است تعداد تعادل نش‌ها بیشتر از یک باشد. اما با استفاده از ابزارهای ریاضی مناسب، می‌توان به راحتی تعادل نش را برای بسیاری از بازی‌ها محاسبه کرد و درک بهتری از رفتار بازیکنان در شرایط مختلف به دست آورد.


