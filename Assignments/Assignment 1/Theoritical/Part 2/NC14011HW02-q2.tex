سوال ۲

\subsection*{الف}

برای نشان دادن اینکه ارزش بازی برابر با صفر است، می‌توانیم نشان دهیم که برای هر استراتژی ترکیبی که نفر اول انتخاب می‌کند، نفر دوم می‌تواند یک استراتژی انتخاب کند که امتیاز نفر اول را صفر کند. سپس نشان می‌دهیم که برای هر استراتژی ترکیبی که نفر دوم انتخاب می‌کند، نفر اول می‌تواند استراتژی‌ای انتخاب کند که امتیاز نفر دوم را صفر کند.

اگر نفر اول راس i را انتخاب کند، نفر دوم می‌تواند راس i را انتخاب کند و بنابراین امتیاز نفر اول برابر با
$W_{i} - w_{ii}$
خواهد بود. اما طبق تعریف
$W_i = \sum_{j=1} ^{n} w_{ij}$
، بنابراین
$W_{i} - w_{ii} = \sum_{j=1, j \neq i} ^{n} w_{ij}$
که مقدار نامنفی است. بنابراین، نفر دوم می‌تواند استراتژی‌ای انتخاب کند که امتیاز نفر اول را صفر یا کمتر از صفر کند.

اگر نفر دوم راس i را انتخاب کند، نفر اول می‌تواند هر راس j را انتخاب کند به طوری که
$w_{ji} = 0$
 . در این حالت، امتیاز نفر اول برابر با صفر خواهد بود. بنابراین، نفر اول می‌تواند استراتژی‌ای انتخاب کند که امتیاز نفر دوم را صفر یا کمتر از صفر کند.

بنابراین، ارزش بازی برابر با صفر است.

\subsection*{ب}

به طور مشابه، می‌توان نشان داد که بردار x وجود دارد که درایه‌های آن نامنفی هستند و مجموع درایه‌های آن برابر یک است و
$x^{T}A = 0$
.

به طور خاص، می‌توانیم بردار x را به گونه‌ای انتخاب کنیم که
$x_i$
برابر با
$\frac{1}{n}$
برای هر i از 1 تا n باشد. در این حالت، می‌توان نشان داد که
$x^{T}A = 0$
.

به این صورت که:
\begin{align*}
	x^{T}A &= \sum_{i=1}^{n} x_i a_{ij} \\
	&= \sum_{i=1}^{n} (1/n) * (w_{ij} – 1_{i = j} * W_{i}) \\
	&= \sum_{i=1}^{n} w_{ij}/n - \sum_{i=1}^{n} 1_{i = j} * W_{i}/n \\
	&= \sum_{i=1}^{n} w_{ij}/n - W_{j}/n \\
	&= W_{j}/n - W_{j}/n \\
	&= 0
\end{align*}
بنابراین، بردار x وجود دارد به طوری که درایه‌های آن نامنفی هستند و مجموع درایه‌های آن برابر یک است و
$x^{T}A = 0$
 .
 
 