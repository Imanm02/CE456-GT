سوال ۳

برای پیدا کردن تعادل نش در بازی دو نفره، ما ابتدا باید بازی را به صورت یک جدول سودمندی نمایش دهیم. در این بازی، مجموعه اعمال ممکن برای هر بازیکن عدد انتخابی از مجموعه 1 تا k است. بنابراین، جدول سودمندی بازی به صورت یک جدول
$k \times k$
خواهد بود.

سودمندی هر بازیکن را می‌توان با توجه به این که آیا اعداد انتخابی توسط آن‌ها با هم برابر است یا خیر تعیین کرد. اگر حمید و مجید عدد یکسانی را انتخاب کنند، مجید به حمید 1 ریال می‌دهد. در این صورت سودمندی حمید 1 خواهد بود و سودمندی مجید 
$-1$
خواهد بود. در غیر این صورت، سودمندی هر دو بازیکن 0 خواهد بود.

یک استراتژی ترکیبی برای یک بازیکن استراتژی‌ای است که بازیکن در آن به جای انتخاب یک عمل خاص، احتمالاتی را برای انتخاب هر عمل مشخص می‌کند. بنابراین، برای این بازی، یک استراتژی ترکیبی به این معنی است که حمید و مجید احتمالاتی را برای انتخاب هر عدد از 1 تا k مشخص می‌کنند.

اگر استراتژی ترکیبی حمید و مجید به ترتیب p و q باشد، سودمندی حمید
$E_p(q)$
و سودمندی مجید
$E_q(p)$
خواهد بود. این بازی یک تعادل نش دارد اگر و فقط اگر هیچ یک از بازیکنان نتوانند سودمندی خود را با تغییر استراتژی خود افزایش دهند. به عبارت دیگر، اگر 
$p*$
و
$q*$
تعادل نش باشد، آنگاه برای هر استراتژی p و q داریم:

$E_p(q*) \leq E_p*(q*)$ 
و
$E_q(p*) \leq E_q*(p*)$

در این بازی، تعادل نش وجود دارد اگر و فقط اگر
$p*$
و
$q*$
به گونه‌ای باشند که حمید و مجید به صورت تصادفی عددی از 1 تا k انتخاب می‌کنند. به عبارت دیگر، اگر
$p* = (\frac{1}{k}, \frac{1}{k}, ..., \frac{1}{k})$
و
$q* = (\frac{1}{k}, \frac{1}{k}, ..., \frac{1}{k})$
باشد. در این صورت، سودمندی انتظاری هر دو بازیکن برابر با 0 خواهد بود و هیچ یک از آن‌ها نمی‌توانند سودمندی خود را با تغییر استراتژی خود افزایش دهند.

می‌توان نشان داد که این تعادل تنها تعادل نش بازی است. اگر یک استراتژی تعادل نش دیگر وجود داشته باشد، باید بتوان سودمندی انتظاری حمید یا مجید را افزایش داد. اما، اگر حمید یا مجید احتمال بیشتری را برای انتخاب یک عدد خاص قرار دهند، بازیکن دیگر می‌تواند با انتخاب همان عدد سودمندی خود را افزایش دهد. بنابراین، تنها تعادل نش که در این بازی وجود دارد تعادلی است که در آن هر دو بازیکن به صورت تصادفی عددی از 1 تا k انتخاب می‌کنند.
