سوال ۴

\subsection*{الف}

\begin{center}
	\begin{array}{c|ccccc}
		& 1 & 2 & 3 & 4 & 5 \\
		\hline
		1 & (0 , 0) & (1 , 2) & (1 , 3) & (1 , 4) & (1 , 5) \\
		2 & (2 , 1) & (0 , 0) & (2 , 3) & (2 , 4) & (2 , 5) \\
		3 & (3 , 1) & (3 , 2) & (0 , 0) & (3 , 4) & (3 , 5) \\
		4 & (4 , 1) & (4 , 2) & (4 , 3) & (0 , 0) & (4 , 5) \\
		5 & (5 , 1) & (5 , 2) & (5 , 3) & (5 , 4) & (0 , 0) \\
	\end{array}
\end{center}

با مغلوب شدن استراتژی‌ها نهایتا به این جدول می‌رسیم:

\begin{center}
	\begin{array}{c|ccc}
		& 3 & 4 & 5 \\
		\hline
		3 & (0 , 0) & (3 , 4) & (3 , 5) \\
		4 & (4 , 3) & (0 , 0) & (4 , 5) \\
		5 & (5 , 3) & (5 , 4) & (0 , 0) \\
	\end{array}
\end{center}



توجه کنید که در یک تعادل متقارن، هر بازیکن با احتمال برابر بین چند عدد متفاوت انتخاب می‌کند. اگر بازیکنی عدد 1 یا 2 را انتخاب کند، بازیکن دیگر می‌تواند با انتخاب یک عدد بزرگتر از آن، امتیاز بیشتری کسب کند. بنابراین، در تعادل متقارن، این دو عدد با احتمال ناصفر بازی نمی‌شوند. به طور مشابه، اگر بازیکنی عدد 5 را انتخاب کند و بازیکن دیگر عددی کوچکتر انتخاب کند، امتیاز آن کمتر خواهد بود. بنابراین، در تعادل، عدد 5 نیز با احتمال ناصفر بازی نمی‌شود. پس تعادل متقارن در این بازی به صورتی است که هر بازیکن با احتمال برابر بین اعداد 3 و 4 انتخاب می‌کند.

\subsection*{ب}

حال می‌خواهیم نشان دهیم که اعداد 1 تا C در هیچ کدام از تعادل‌ها با احتمال ناصفر بازی نمی‌شوند، که C برابر است با
$$
C = [ n - \frac {\sqrt{8n + 1} - 1} {2} ]
$$

در تعادل متقارن، اگر بازیکنی عدد i را انتخاب کند (که i کمتر از C است)، بازیکن دیگر می‌تواند با انتخاب عدد j (که j بیشتر از i و کمتر از n است)، امتیاز بیشتری کسب کند. پس اعداد کمتر از C در هیچ کدام از تعادل‌ها با احتمال ناصفر بازی نمی‌شوند.

برای تشخیص چرا مقدار C دقیقاً به شکل نشان داده شده تعریف می‌شود، ابتدا باید به طرز کلی بازی را در نظر بگیریم. در بازی ما، هر بازیکن در هر دور از بازی یکی از اعداد 1 تا n را انتخاب می‌کند. اگر دو بازیکن عدد متفاوتی انتخاب کنند، بازیکنی که عدد بزرگ‌تر را انتخاب کرده است، امتیاز بیشتری می‌گیرد.

اگر بازیکنی عددی کوچکتر از C را انتخاب کند، بازیکن دیگر می‌تواند با انتخاب عددی بزرگتر از عدد انتخاب شده (اما کوچکتر از n )، امتیاز بیشتری کسب کند. بنابراین، بهترین استراتژی برای هر بازیکن این است که عددی را انتخاب کند که حداقل برابر با C باشد.

پس برای یافتن مقدار دقیق C ، باید یک معادله را حل کنیم که نشان می‌دهد که انتخاب هر عدد کوچکتر از C امتیاز کمتری نسبت به انتخاب هر عدد بزرگتر از آن می‌دهد. این معادله به شکل زیر است:

$$
\frac{C (C + 1)}{2} = \frac{n (n + 1)}{2} - \frac{C (C + 1)}{2}
$$

این معادله بیان می‌کند که مجموع اعداد از 1 تا C برابر است با تفاوت مجموع اعداد از 1 تا n و مجموع اعداد از 1 تا C .

حالا با حل معادله می‌رسیم به:

$$
C = [ n - \frac {\sqrt{8n + 1} - 1} {2} ]
$$

از این جا می‌توانیم ببینیم که اگر هر بازیکن عددی را انتخاب کند که کمتر از C است، بازیکن دیگر همیشه می‌تواند با انتخاب عددی بزرگتر، امتیاز بیشتری کسب کند. پس استراتژی بهینه برای هر بازیکن این است که عددی را انتخاب کند که حداقل برابر با C باشد.
