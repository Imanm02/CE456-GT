سوال ۱


در این سوال برای هر کدوم از بازی‌ها، تعادل‌های نش رو با حذف استراتژی‌های مغلوب و مشخص کردن بهترین پاسخ هر استراتژی حریف پیدا می‌کنیم.

\subsection*{الف}

در این بخش از دو جدولی که داریم، زمانی که بهترین پاسخ‌ها رو مشخص کنیم، هیچ نقطه‌ای از جدول نیست که هر ۳ عدد آن بهترین پاسخ باشند در نتیجه تعادل نش خالص نداریم.

پس احتمال برای هر کدام فرض می‌کنیم و به محاسبه‌ی 
\lr{Mixed Strategy Nash Equilibrium}
می‌پردازیم.

اگر بازیکن سمت چپ، با احتمال p استراتژی T را بازی کند و بازیکن سمت بالا، با احتمال q استراتژی L را بازی کند و بازیکن سوم استراتژی X را با احتمال k بازی کند، خواهیم داشت:

$$
E(U_1) = 2qk + 2q - 5k + 1 = 3qk - 4q + 2k
$$

$$
\rightarrow k = \frac{6p + 1}{p + 7}
$$

$$
E(U_2) = 2pk + 2p - 5k + 1 = 3pk - 4p + 2k
$$

$$
\rightarrow k = \frac{6q + 1}{q + 7}
$$

$$
E(U_3) = -6pq + 4q + 4p - 2 = -6pq + 2q + 2p
$$

$$
\rightarrow p = 1 - q
$$

از اون‌جایی که مقدار p و q در بخش‌های مختلف معادله مثل این‌که فقط جای‌شان فرق کرده است پس داریم:

$$
p = q = \frac{1}{2}
$$

و با توجه به این دو بدست می‌آوریم:

$$
r = \frac{8}{15}
$$

پس برای تعادل نش خواهیم داشت:

$$
{(\frac{1}{2} , \frac{1}{2}) , (\frac{1}{2} , \frac{1}{2}) , (\frac{8}{15} , \frac{7}{15})}
$$

\subsection*{ب}

در ابتدا برای هر کدام از استراتژی‌های X و Y و Z احتمال در نظر می‌گیریم.

احتمال X را p در نظر می‌گیریم.

احتمال Y را q در نظر می‌گیریم.

احتمال Z را 
$1 - q - p$
در نظر می‌گیریم.

سپس برای هر کدام از استراتژی‌های A و B و C احتمال در نظر می‌گیریم.

احتمال A را m در نظر می‌گیریم.

احتمال B را n در نظر می‌گیریم.

احتمال C را 
$1 - m - n$
در نظر می‌گیریم.

حالا سعی می‌کنیم استراتژی‌های مغلوب را حذف کنیم و بین باقی استراتژی‌های به دنبال تعادل‌های ترکیبی بگرذیم.

استراتژی A توسط استراتژی‌های B و C مغلوب است، پس به حذف آن می‌پردازیم و عملیات‌ها را بین باقی استراتژی‌ها انجام می‌دهیم.

حالا تساوی‌ها را برای استراتژی‌های مختلف می‌نویسیم و مقدار هر کدام از احتمال‌ها را با توجه به تساوی‌ها بدست می‌آوریم:

$$
p \times 0 + q \times 2 + (1 - p - q) \times 4 = p \times 0 + q \times 1 + (1 - p - q) \times 3
$$

$$
\rightarrow p = 1 , q = 0
$$

$$
n \times -1 + (1 - n) \times 4 = n \times 2 + 3 \times 1 - n = 2n + 3 - 3n
$$

$$
\rightarrow n = \frac{1}{4}
$$

برای تعادل نش داریم:

$$
( (1, 0, 0) , (1, \frac{1}{4} , \frac{3}{4}))
$$

با توجه به این‌که میزان پوینت بازیکن چپ اگر با استراتژی Y یا Z بازی کند، بیشتر از استراتژی X می‌شود با توجه به مقادیز بدست اومده، با حذف استراتژی X و سپس مغلوب شدن استراتژی B در جدول جدید، برای تعادل نش نهایی به این مقدار خواهیم رسید:

مقدار n همواره از 
$\frac{1}{4}$
بیشتر می‌شود،
پس داریم برای تعادل نش:

$$
((0 , q, 1 - q) , (0 , n , 1 - n))
$$

مقادیر q بین ۰ و ۱ ، مقادیر n نیز بین 
$\frac{1}{4}$
و ۱ می‌باشد.


