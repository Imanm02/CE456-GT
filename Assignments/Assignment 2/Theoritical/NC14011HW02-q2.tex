در این سوال و با توجه به این‌که بازی بی‌نهایت دور تکرار می‌شود، برای پیدا کردن تعادل نش، به بهترین جواب‌ها مراجعه می‌کنیم.


حالا داریم که بازیکن سمت چپ هیچ‌گاه A بازی نمی‌کند چون هر دو مقدار پاداش وی از حالت بازی B کمتر است و همچنین بازیکن سمت راست هیچ‌گاه B بازی نمی‌کند چون هر دو مقدار پاداش وی از حالت بازی A کمتر است.


در نتیجه نقاط
$(A, A)$
و
$(B, B)$
هیچ‌گاه تعادل نش نخواهند بود.


حالا چون بازی تکرارشونده است، پس افراد می‌توانند با توافق، سود بیشتری داشته باشند.


پس برای مثال می‌توانند توافق کنند که هر دو دور در کل ۵ تا سود ببرند و در واقع هر دور یکی ۵ سود ببرد و یکی ۰ و در نتیجه نسبت به حالت تعادل نش غیرتکرارشونده، سود بیشتری خواهند داشت.


اما با توجه به این‌که discount در این‌جا مطرح است، اگر نرخ تخفیف کم باشد، مشخص نیست که همچین تصمیمی در آینده بگیریم و ممکن است تصمیم ما تغییر کند و در واقع همیشه این تصمیم درست است در صورتی که همین میزان پاداش را همواره داشته باشیم و در بلند مدت تغییری نداشته باشیم در این میزان از پاداش و نباید در این یک دور در میان، نرخ تحفیف باعث ضرر ما شود.

در نتیجه در صورتی که نرخ تخفیف بالا و نزدیک به ۱ باشد، حالت ایده‌آل سوال رخ می‌دهد.

اما برای بدست آوردن مقدار دقیق تخفیف، باید سیاست جریمه‌ی grim را بررسی کنیم چون اگر جواب برای این سیاست درست باشد، برای دیگر سیاست‌ها درست است قطعا.

سیاست grim یک استراتژی برای بازی‌های تکرارشونده است که در آن بازیکن در ابتدا همکاری می‌کند و همکاری را ادامه می‌دهد تا زمانی که بازیکن دیگر خیانت کند. در این نقطه، بازیکن grim هرگز دوباره همکاری نمی‌کند.

با نوشتن میزان payoff های بازیکنان و مقایسه‌ی آن‌ها با توجه به 
$\delta$
که میزان discount است، بدست می‌آوریم:

$$
5 \frac{\delta^{2k-1}}{1 - \delta^2} \geq \frac{\delta^{2k}}{1 - \delta} 
$$

$$
\rightarrow \delta \geq \frac{1}{4}
$$