تمام استراتژی‌های پایدار تکاملی، در استراتژی‌های خالص و ترکیبی بازی زیر را پیدا می‌کنیم.

\begin{center}
	\begin{array}{c|ccc}
		& & & \\ \hline
		& (0 , 0) & (3 , 1) & (0 , 0) \\
		& (1 , 3) & (0 , 0) & (0 , 0) \\
		& (0 , 0) & (0 , 0) & (1 , 1) \\
	\end{array}
\end{center}	


ابتدا استراتژی‌های خالص را بررسی می‌کنیم. در یک بازی به صورت عمومی، برای اینکه استراتژی i یک استراتژی پایدار تکاملی باشد، باید دو شرط را بررسی کنیم:

استراتژی i باید برنده استراتژی j را وقتی بقیه بازیکنان استراتژی i را انتخاب کرده باشند، برابر یا بیشتر از استراتژی j کند.
اگر استراتژی i و j هر دو به یک اندازه برنده باشند، آنگاه استراتژی i باید برنده استراتژی j را وقتی بقیه بازیکنان استراتژی j را انتخاب کرده باشند، بیشتر از استراتژی j کند.
در مثال شما، به نظر می‌رسد استراتژی‌های 1 و 2 برای هر دو بازیکن، پاداش بیشتری نسبت به استراتژی 3 ارائه می‌دهند. بنابراین، می‌توان گفت استراتژی 3 پایدار تکاملی نیست.

از طرفی، استراتژی 1 برای بازیکن اول و استراتژی 2 برای بازیکن دوم، پاداش بیشتری نسبت به استراتژی‌های دیگر ارائه می‌دهند. بنابراین، می‌توان گفت استراتژی 1 برای بازیکن اول و استراتژی 2 برای بازیکن دوم، پایدار تکاملی هستند.

در مورد استراتژی‌های ترکیبی، چون پاداش‌ها به صورت اعداد طبیعی ارائه شده‌اند و هیچ استراتژی ترکیبی با پاداش بیشتری وجود ندارد، می‌توان گفت هیچ استراتژی ترکیبی پایدار تکاملی وجود ندارد.

به همین ترتیب، می‌توان گفت تنها استراتژی‌های پایدار تکاملی در این بازی، استراتژی 1 برای بازیکن اول و استراتژی 2 برای بازیکن دوم هستند.