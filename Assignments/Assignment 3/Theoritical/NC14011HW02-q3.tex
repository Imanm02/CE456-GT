با توجه به ماتریس بازده، می‌توانیم مشاهده کنیم که هیچ تعادلی در این بازی وجود ندارد که هر دو بازیکن از انتخاب همان استراتژی سود ببرند، یا به عبارت دیگر، تعادل نش در این بازی وجود ندارد. برای نشان دادن این موضوع، می‌توانیم به این نکته اشاره کنیم که برای وجود تعادل نش در یک بازی، باید بتوانیم یک انتخابی پیدا کنیم که برای هر دو بازیکن، سود بهتری نسبت به هر انتخاب دیگری داشته باشد. با توجه به ماتریس بازده ارائه شده، چنین انتخابی وجود ندارد.

با این حال، بازی ممکن است تعادل همبسته داشته باشد. تعادل همبسته یک تعادل است که در آن بازیکنان استراتژی‌های تصادفی انتخاب می‌کنند. برای پیدا کردن تعادل همبسته، باید سعی کنیم مقداری را پیدا کنیم که با استفاده از آن، بازیکنان به طور تصادفی بین استراتژی‌های مختلف انتخاب کنند. در این مثال، یک تعادل همبسته به شرح زیر است:

برای بازیکن اول:

$$
(p_1, p_2, p_3) = ( \frac{1}{3} , \frac{1}{3} , \frac{1}{3} )
$$

برای بازیکن دوم:

$$
(q_1, q_2, q_3) = ( \frac{1}{3} , \frac{1}{3} , \frac{1}{3} )
$$

این استراتژی‌های تصادفی باعث می‌شوند که هر دو بازیکن بتوانند از بازی سود ببرند. به این ترتیب، تعادل همبسته پیدا می‌شود.