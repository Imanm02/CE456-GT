یک بازی در تعادل تکاملی
\lr{(Evolutionary Stable Strategy or ESS)}
است اگر یک استراتژی پیدا شود که اگر تمام جمعیت از آن استفاده کند، هیچ استراتژی دیگری نتواند آن را در حالت مستقر شکست دهد. به عبارت دیگر، اگر تمام جمعیت از استراتژی ESS استفاده کند، هیچ استراتژی دیگری نمی‌تواند در برابر آن پیروز شود.

برای این بازی، می‌توانیم از شرایط تعادل تکاملی استفاده کنیم. یک استراتژی ESS ، استراتژی i است اگر و تنها اگر:

$$
U(i, i) \geq U(j, i) \hspace{1.5mm} for \hspace{1.5mm} each \hspace{1.5mm} j \neq i.
$$

این به این معنی است که یک استراتژی باید حداقل به اندازه هر استراتژی دیگر عملکرد خوبی داشته باشد وقتی که بازیکنان دیگر از همان استراتژی استفاده می‌کنند.

اگر
$$U(j, i) = U(i, i)$$
برای بعضی
$$j \neq i$$
، آنگاه
$$U(i, j) > U(j, j)$$

این به این معنی است که اگر استراتژی دیگری عملکرد مشابهی نسبت به استراتژی ESS داشته باشد، بازیکن با استفاده از استراتژی ESS در مقابل استراتژی دیگر بهتر عمل می‌کند.

با استفاده از این دو شرط، می‌توانیم شرایط لازم برای x را پیدا کنیم. برای استراتژی 1 به عنوان استراتژی ESS ، می‌گیریم:

$$U(1, 1) ≥ U(2, 1) \hspace{1.5mm} means \hspace{1.5mm} x \geq 0.5$$

و اگر
$$x = 0.5$$
آنگاه باید داشته باشیم
$$U(1, 2) > U(2, 2)$$
یعنی
$$0.5 > 1$$
که این ناقض است.

بنابراین استراتژی 1 هنگامی استراتژی تکاملی مستقر است که
$$x > 0.5$$


برای استراتژی 2 به عنوان استراتژی ESS ، می‌گیریم:

$$U(2, 2) \geq U(1, 2) \hspace{1.5mm} means \hspace{1.5mm} 1 \geq 0.5x$$

و اگر
$$1 = 0.5x$$
آنگاه باید داشته باشیم
$$U(2, 1) > U(1, 1) \hspace{1.5mm} means \hspace{1.5mm} 0.5 > x $$
که این ناقض است.

بنابراین استراتژی 2 همیشه استراتژی تکاملی مستقر است.

به نظر می‌رسد که در این بازی، استراتژی 2 همیشه ESS است و استراتژی 1 فقط زمانی ESS است که
$$x > 0.5$$

بنابراین بازی دارای تعادل تکاملی است هنگامی که
$$x ≤ 0.5 \hspace{1.5mm} or \hspace{1.5mm} x > 0.5$$

