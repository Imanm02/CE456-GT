\subsection*{الف}


X : 4 رأی از گروه A


Y : 2 رأی از گروه C


Z : 3 رأی از گروه B


بنابراین در مرحله اول، گزینه Y با کمترین رأی (2 رأی) حذف می‌شود.


حالا در مرحله دوم، فقط دو گزینه X و Z باقی مانده‌اند. باید دوباره تعداد رأی‌ها را بشماریم با توجه به اینکه گزینه Y حذف شده است.


گروه A هنوز X را به Z ترجیح می‌دهد و 4 رأی به X می‌دهد.


گروه B هنوز Z را به X ترجیح می‌دهد و 3 رأی به Z می‌دهد.


گروه C حالا بر اساس اولویت دوم، Z را به X ترجیح می‌دهد و 2 رأی به Z می‌دهد.


با توجه به اینکه گروه A چهار نفره است و نمی‌تواند در مرحله اول رای گیری با اکثریت رای برنده شود، باید استراتژی خود را تغییر دهد تا در مرحله دوم انتخاب مورد نظر خود را کسب کند.


گروه A می‌داند که اگر در مرحله اول گزینه X را انتخاب کند، با توجه به اینکه گروه B و C با هم 5 رای دارند و در مرحله دوم به گزینه Z رای می‌دهند، گزینه X حذف خواهد شد. پس گروه A برای جلوگیری از حذف گزینه X باید در مرحله اول به گزینه Y رای دهد تا گزینه Z حذف شود.


بنابراین، گروه A باید در مرحله اول به گزینه Y رای دهد. در این صورت، تعداد رای‌ها به شرح زیر خواهد بود:


X : 0 رأی


Y : 4 رأی از گروه A و 2 رأی از گروه C. مجموعاً 6 رأی


Z : 3 رأی از گروه B


در نتیجه، گزینه X با کمترین رأی حذف می‌شود. در مرحله دوم، با حذف گزینه X، گروه A می‌تواند به گزینه مورد علاقه خود یعنی Y رای دهد و این باعث می‌شود گزینه Y با 6 رأی برنده شود. پس گروه A برای اینکه گزینه مورد علاقه‌ی خود را در مرحله دوم انتخاب کند، باید در مرحله اول به گزینه Y رای دهد.


در نهایت، X چهار رأی و Z پنج رأی خواهد داشت. پس گزینه نهایی که انتخاب می‌شود، Z یعنی « عدم تغییر مدیرعامل و افزایش حقوق او » است.