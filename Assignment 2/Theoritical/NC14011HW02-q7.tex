\subsection*{الف}

در این بازی دو بازیکن داریم و هر بازیکن دو انتخاب دارد، که به ترتیب با حروف $C$ و $D$ نشان داده شده‌اند. بازده‌های بازیکنان در هر وضعیت بازی، زوج اعدادی است که در خانه‌های جدول آمده‌اند.

تعادل نش یکی از مفاهیم مهم در نظریه بازی‌ها است و به موقعیتی اشاره می‌کند که هیچ یک از بازیکنان سودی در تغییر استراتژی خود ندارند، به شرطی که بازیکنان دیگر استراتژی خود را تغییر ندهند.

برای یافتن تعادل نش در این بازی، باید به ترتیب بهره‌وری هر استراتژی را برای هر بازیکن در نظر گرفت:

اگر بازیکن اول $C$ را انتخاب کند، بازیکن دوم برای حداکثر سود، استراتژی $D$ را انتخاب می‌کند (زیرا $6 > 4$).

اگر بازیکن اول $D$ را انتخاب کند، بازیکن دوم دوباره برای حداکثر سود، استراتژی $D$ را انتخاب می‌کند (زیرا $1 > 0$).

پس بازیکن دوم همیشه استراتژی $D$ را انتخاب می‌کند.

حالا اگر بازیکن دوم $D$ را انتخاب کند، بازیکن اول برای حداکثر سود، استراتژی $D$ را انتخاب می‌کند (زیرا $6 > 0$).
پس تنها تعادل نش این بازی، وقتی هر دو بازیکن استراتژی $D$ را انتخاب می‌کنند، رخ می‌دهد.

\subsection*{ب}

استراتژی توضیح داده شده، یک نوع از استراتژی‌های متقابل تکرار (Tit-for-Tat) است و در بازی‌های تکرارشونده بسیار متداول است. در این استراتژی، هر دو بازیکن در ابتدا با هم همکاری می‌کنند و اگر یک بازیکن تخلف کند، بازیکن دیگر هم برای مدتی تعیین شده از همکاری منصرف می‌شود.

در اینجا، ما می‌خواهیم ببینیم چه مقدار n باعث می‌شود تا این استراتژی تعادل نش زیربازی کامل داشته باشد. برای اینکه این استراتژی تعادل باشد، بازیکنان نباید سودی در تخلف کردن از استراتژی داشته باشند.

اگر یک بازیکن در یک دور C را انتخاب کند، اما سپس تخلف کند و D را انتخاب کند، او در آن دور 6 واحد بهره می‌برد در مقایسه با 4 واحد که اگر C را انتخاب کرده بود می‌گرفت. اما سپس او برای n دور بعدی باید 1 واحد بهره بگیرد به جای 4 واحد که اگر همچنان در استراتژی مانده بود می‌گرفت.

به طور خلاصه، برای اینکه این استراتژی تعادل باشد، باید داشت:

$$
-2 + 3 \delta + ... + 3 \delta^n \geq 0 \rightarrow n \geq 2
$$
