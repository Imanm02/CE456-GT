\subsection*{تعریف}

بازی های بیزین یا
\lr{Bayesian Games}
، یک دسته از نظریه بازی است که در آن بازیکنان دارای اطلاعات ناقصی درباره‌ی سایر بازیکنان هستند. در واقع, در این نوع بازی, هر بازیکن از مجموعه‌ی خاصی از نوع‌ها 
$(types)$ 
است که اطلاعات او را درباره‌ی محیط بازی تعیین می‌کند. این اطلاعات می‌تواند شامل مواردی مانند استراتژی، توانایی‌ها، هدف و یا معیارهای سود بازیکن باشد. یک ایده مهم در بازی های بیزین این است که بازیکنان بر اساس احتمال نوع دیگر بازیکنان رفتار می‌کنند. در نتیجه, تصمیمات بازیکنان در این بازی‌ها نه تنها بر اساس انتظارات خود از رفتار سایر بازیکنان است, بلکه بر اساس دانش خود از توزیع احتمالاتی انواع دیگر بازیکنان نیز است.

\subsection*{مثال}

مثال 
\lr{Sheriff’s Dilemma}
یک بازی بیزینی است که در آن یک شریف تصمیم می‌گیرد که یک مظنون را بازجو کند یا نکند، در حالی که مظنون می‌تواند بی‌گناه یا مجرم باشد. در این مثال، انواع مختلف
$(types)$ 
برای مظنون وجود دارد: بی‌گناه و مجرم. اطلاعات شریف در مورد نوع مظنون محدود است و او تنها می‌تواند با توجه به توزیع احتمالاتی این انواع رفتار کند.

شریف دو انتخاب دارد: بازجویی کردن یا عدم بازجویی. اگر شریف یک بی‌گناه را بازجو کند، هزینه‌ای برای خود و ضرری برای بی‌گناه به وجود می‌آورد. اگر شریف یک مجرم را بازجو کند، مجازات مجرم می‌شود که یک پاداش برای شریف به حساب می‌آید.


بازی به این ترتیب ادامه می‌یابد که شریف تصمیم می‌گیرد که چه کاری انجام دهد، با توجه به توزیع احتمالاتی انواع مظنون. این یک بازی بیزینی است چون اطلاعات شریف درباره نوع مظنون محدود است و باید براساس توزیع احتمالاتی این انواع تصمیم بگیرد.

\subsection*{تعریف ریاضی}

یک بازی بیزین به صورت ریاضی شامل چهار عنصر است: مجموعه بازیکنان، مجموعه نوع بازیکنان، مجموعه استراتژی‌ها و تابع پاداش.


\textbf{مجموعه بازیکنان:} $N$ بازیکن وجود دارد که با $\{1, 2, ..., N\}$ نمایش داده می‌شوند.

\textbf{مجموعه نوع بازیکنان:} برای هر بازیکن $i$ در مجموعه بازیکنان، مجموعه $T_i$ نوع‌های ممکن برای بازیکن $i$ را نشان می‌دهد. هر نوع $t$ از $T_i$ برای بازیکن $i$ احتمال $p(t)$ را دارد.

\textbf{مجموعه استراتژی‌ها:} برای هر نوع $t$ از بازیکن $i$، $S_i(t)$ مجموعه استراتژی‌های ممکن برای بازیکن $i$ را نشان می‌دهد.

\textbf{تابع پاداش:} برای هر نوع $t$ از بازیکن $i$ و استراتژی $s$ در $S_i(t)$، $u_i(s, t)$ پاداش بازیکن $i$ را نشان می‌دهد.

در بازی بیزین، هر بازیکن فقط نوع خود را می‌شناسد و بر اساس این اطلاعات و احتمال نوع سایر بازیکنان، استراتژی را انتخاب می‌کند. یک تعادل بیزین نش
\lr{Bayesian Nash Equilibrium}
وجود دارد اگر هیچ بازیکنی برای تغییر استراتژی خود (با توجه به احتمالاتی که برای انواع سایر بازیکنان در نظر گرفته) منفعتی نداشته باشد.


\subsection*{یک تعریف ریاضی دیگر}


در تعریف ریاضی بازی‌های بیزین در نظریه بازی، از عناصر زیر استفاده می‌شود:

\textbf{مجموعه بازیکنان}:
 \lr{N}
مجموعه‌ای از بازیکنان که در بازی حضور دارند. هر بازیکن با یک شماره شناسایی مشخص می‌شود.

\textbf{مجموعه استراتژی‌ها}:
 \lr{A}
برای هر بازیکن، مجموعه‌ای از استراتژی‌ها وجود دارد که بازیکن می‌تواند انتخاب کند. استراتژی‌ها می‌توانند شامل تصمیم‌ها، اقدامات، یا رفتارهای مختلفی باشند.

\textbf{فضای فرضیه‌ها}:
\lr{$\theta$}
یک مجموعه از فرضیه‌های احتمالاتی که بازیکنان درباره تصمیمات و استراتژی‌های همدیگر دارند. هر بازیکن یک توزیع احتمالی را برای فرضیات خود شکل می‌دهد. فرضیه‌ها می‌توانند مربوط به تصمیمات گذشته، تصمیمات فعلی یا تصمیمات آینده باشند.

\textbf{تابع پاداش}:
\lr{p}
یک تابع که به هر بازیکن و هر دسته از استراتژی‌ها مقدار پاداش را نسبت می‌دهد. این تابع می‌تواند بر اساس ترکیبی از تصمیمات بازیکنان و فرضیه‌های احتمالاتی آن‌ها تعریف شود.

\textbf{تابع انتفاع}:
\lr{u}
یک تابع که به هر بازیکن و هر دسته از استراتژی‌ها انتفاع را نسبت می‌دهد. این تابع معمولاً ترجیحات بازیکنان را نمایش می‌دهد و نشان می‌دهد که بازیکنان به کدام استراتژی‌ها بیشتر علاقه دارند.

با استفاده از این پنج عنصر، می‌توان بازی‌های بیزین را به صورت ریاضی مدل کرده و تحلیل کرد.


\subsection*{نوع یا type}

در بازی‌های بیزین، «نوع» یک بازیکن عموماً اشاره به اطلاعاتی دارد که او درباره بازی می‌داند. همانطور که پیش‌تر ذکر کردیم، این ممکن است شامل اطلاعات درباره استراتژی، توانایی‌ها، هدف یا معیارهای سود بازیکن باشد. اما در برخی از بازی‌ها، «نوع» یک بازیکن ممکن است شامل اطلاعات بیشتری باشد، از جمله اطلاعات درباره دانش یا اعتقادات سایر بازیکنان. در این حالت، نوع یک بازیکن را می‌توان به عنوان یک "نوع آگاهانه» یا 
\lr{epistemic type}
توصیف کرد.

در این بردار، «نوع آگاهانه» یک بازیکن شامل دو بخش است: یک بخش که اطلاعات شخصی یا خصوصیات بازیکن را نشان می‌دهد (مانند توانایی‌های خود و استراتژی‌هایی که او می‌تواند انتخاب کند)، و بخش دیگر که دانش یا باورهای بازیکن را درباره نوع سایر بازیکنان نشان می‌دهد. این دانش می‌تواند شامل اطلاعات درباره استراتژی‌هایی باشد که سایر بازیکنان احتمالاً انتخاب می‌کنند، یا حتی اطلاعات درباره باورهای سایر بازیکنان در مورد بازیکن اولیه.

\subsection*{تعادل بیزین نش}

با استفاده از این توصیفات، می‌توانیم مفهوم پیچیده‌تری از تعادل بیزین نش در بازی‌ها با انواع آگاهانه تعریف کنیم. در چنین تعادلی، هیچ بازیکنی منفعتی برای تغییر استراتژی خود ندارد، هنگامی که تمام بازیکنان اطلاعات خود را به درستی می‌فهمند و با توجه به باورهای خود درباره انتخاب‌های سایر بازیکنان عمل می‌کنند.


در تعادل بیزین نش، هر بازیکن به عنوان یک عامل رشدی و هوشمند، با توجه به احتمالاتی که برای انواع سایر بازیکنان در نظر گرفته، بهترین استراتژی را انتخاب می‌کند. به عبارت دیگر، هر بازیکن در تعادل بیزین نش، اطلاعات خود را در مورد نوع خود و توزیع احتمالاتی نوع‌های دیگر بازیکنان استفاده می‌کند تا استراتژی بهینه خود را تعیین کند.


برای تعیین تعادل بیزین نش در یک بازی بیزین، باید بررسی شود که آیا هر بازیکن با توجه به استراتژی‌های دیگر بازیکنان و توزیع احتمالاتی نوع‌ها، تغییری در استراتژی خود صورت نمی‌دهد. این بدین معناست که هیچ بازیکنی تمایلی به تغییر استراتژی خود ندارد زیرا با انتخاب استراتژی جدید، منفعت خود را کاهش می‌دهد.


تعادل بیزین نش می‌تواند در بازی‌های بیزین ساده و پیچیده ایجاد شود و اغلب نیاز به استفاده از روش‌های ریاضی پیچیده دارد. با استفاده از این تعادل، می‌توان رفتار و تصمیماتی را که بازیکنان در یک بازی بیزین در مواجهه با نوع‌های مختلف ممکن اتخاذ می‌کنند، بررسی و تحلیل کرد.


در کل، بازی های بیزین با در نظر گرفتن اطلاعات ناقص و توزیع احتمالاتی انواع، برای مدلسازی و تحلیل مواقع واقعی که بازیکنان در آن‌ها دارای دانش و قدرت تصمیم‌گیری متفاوتی هستند، بسیار مناسب هستند.





