مسئله‌ای که به آن اشاره شده، به نام «مزایده ویکری» یا «مزایده قیمت دوم» شناخته می‌شود. در این مزایده، برنده کسی است که بیشترین پیشنهاد را ارائه می‌دهد اما مبلغی که پرداخت می‌کند برابر است با دومین بالاترین پیشنهاد. این نوع مزایده از دیدگاه «طراحی مکانیزم‌ها» بسیار جذاب است، زیرا ویلیام ویکری اثبات کرده است که در این محیط، ترغیبی برای دروغ گفتن یا مبالغه کردن در میزان پیشنهاد وجود ندارد.

اگر همه‌ی شرکت‌کنندگان بر اساس ارزش واقعی کالا برای خودشان پیشنهاد بدهند، آن‌ها در یک حالت تعادل بازی قرار می‌گیرند. این یک استراتژی ضعیفا غالب است، زیرا اگر بقیه بازیکنان ارزش واقعی خود را پیشنهاد دهند، هر بازیکن هم بهتر است که ارزش واقعی خود را پیشنهاد دهد. بنابراین، تمام بازیکنان بهتر است که راستگو باشند.

برای اثبات این موضوع، فرض کنید که شما در حراجی شرکت می‌کنید و کالا ارزش واقعی $v$ را برای شما دارد. در این حالت دو سناریو وجود دارد:

اگر بیشترین پیشنهاد دیگران کمتر از $v$ باشد، آنگاه شما باید پیشنهاد $v$ بدهید. اگر کمتر پیشنهاد دهید، کالا را از دست می‌دهید، در حالی که ارزش آن برای شما بیشتر است. اگر بیشتر پیشنهاد دهید، هنوز برنده می‌شوید ولی با هزینه‌ی بیشتر، که نیازی به آن نیست.

اگر بیشترین پیشنهاد دیگران بیشتر از $v$ باشد، آنگاه شما نباید بیشتر از $v$ پیشنهاد دهید. اگر کمتر پیشنهاد دهید، کالا را از دست می‌دهید، اما این اشکالی ندارد زیرا قیمت آن بیشتر از ارزش آن برای شما است. اگر بیشتر پیشنهاد دهید، ممکن است برنده شوید، اما با هزینه‌ی بیشتر از ارزش واقعی آن برای شما، که این موضوع منطقی نیست.

بنابراین، در هر دو حالت، بهترین استراتژی این است که پیشنهادی برابر با ارزش واقعی کالا برای شما باشد.

با عنایت به توضیحات بالا، می‌توانیم مثالی بزنیم تا این موضوع روشن‌تر شود.

فرض کنید 3 نفر در یک مزایده قیمت دوم شرکت می‌کنند و ارزش واقعی کالا برای آن‌ها به ترتیب $v_1=100$، $v_2=200$، و $v_3=300$ است.

اگر همه‌ی شرکت‌کنندگان بر اساس ارزش واقعی کالا برای خودشان پیشنهاد بدهند، 


پیشنهادات آن‌ها به ترتیب $b_1=100$، $b_2=200$، و $b_3=300$ خواهد بود. در این صورت، فرد سوم کالا را برنده می‌شود ولی فقط مبلغ $200$ (دومین بیشترین پیشنهاد) را پرداخت می‌کند.

حال فرض کنید که فرد سوم تصمیم می‌گیرد که بیشتر از ارزش واقعی کالا برای خودش پیشنهاد دهد و $b_3=400$ پیشنهاد می‌کند. باز هم او کالا را برنده می‌شود، اما باز هم فقط $200$ پرداخت می‌کند. پس این تغییر استراتژی هیچ سودی برای او نداشت.

اگر فرد سوم تصمیم بگیرد که کمتر از ارزش واقعی کالا برای خودش پیشنهاد دهد و $b_3=150$ پیشنهاد کند، او کالا را از دست می‌دهد، زیرا پیشنهاد دوم (که برابر با $200$ است) بالاتر از پیشنهاد اوست. پس این تغییر استراتژی هم به ضرر او بود.

پس با توجه به تحلیل بالا، می‌توان فهمید که بهترین استراتژی برای هر فرد این است که پیشنهادی برابر با ارزش واقعی کالا برای خودش داشته باشد.


می‌توان این موضوع را با استفاده از نظریه بازی‌ها و به خصوص بازی‌های تعادل نش بصورت دقیق تر اثبات کرد. در اینجا تلاش می‌کنیم تا با استفاده از ریاضیات این موضوع را توضیح دهیم.

معمولا در مزایده قیمت دوم، تعدادی مزایده‌کننده وجود دارد که برای خرید یک کالا مزایده می‌کنند. فرض کنید هر مزایدکننده $i$ ارزشی $v_i$ برای کالا قائل شده‌است و $b_i$ مبلغی است که او برای کالا پیشنهاد می‌دهد.

اگر برنده‌ی مزایده باشد، سود او برابر است با ارزشی که او از کالا می‌گیرد منهای قیمتی که باید پرداخت کند. اگر برنده نباشد، سود او صفر است. بنابراین، اگر $b_{(1)}$ و $b_{(2)}$ بزرگترین و دومین بزرگترین پیشنهاد‌ها باشند، سود مزایدکننده $i$ به صورت زیر تعریف می‌شود:


اگر $b_i = b_{(1)}$ آنگاه:
$$
u_i(b) =
\begin{cases}
	v_i - b_{(2)} &\quad \text{if } b_{(2)} < v_i \text{,} \
	0 &\quad \text{if } b_{(2)} \geq v_i \text{.}
\end{cases}
$$

و اگر $b_i \neq b_{(1)}$ آنگاه $u_i(b) = 0$.


حال می‌خواهیم نشان دهیم که برای مزایده‌کننده $i$، بهترین استراتژی (به عبارتی استراتژی ضعیفا قالب) این است که $b_i = v_i$ را انتخاب کند.

اگر $b_i < v_i$ باشد، ممکن است وضعیتی پیش بیاید که $b_i < b_{(2)} < v_i$. در این حالت، مزایدکننده $i$ کالا را برنده نمی‌شود، در حالی که اگر $b_i = v_i$ را انتخاب می‌کرد، کالا را با قیمت $b_{(2)}$ برنده می‌شد که کمتر از ارزش آن برای اوست. پس $b_i < v_i$ استراتژی مناسبی نیست.

اگر $b_i > v_i$ باشد، ممکن است وضعیتی پیش بیاید که $v_i < b_{(2)} < b_i$. در این حالت، مزایدکننده $i$ کالا را با قیمت $b_{(2)}$ برنده می‌شود، که بیشتر از ارزش آن برای اوست. پس $b_i > v_i$ هم استراتژی مناسبی نیست.

بنابراین، بهترین استراتژی برای مزایدکننده $i$ این است که $b_i = v_i$ را انتخاب کند. این استراتژی ضعیفا قالب است، چون هرچه باقی مزایدکنندگان چه کنند، مزایدکننده $i$ سودی بیشتری نمی‌تواند ببرد.